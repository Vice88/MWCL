%%%%%%%%%%%%%%%%%%%%%%%%%%%%%%%%%%%%%%%%%%%%%%%%%%%%%%%%%%%%%%%%%%%%%%%%%%%%%%%

\documentclass[journal]{IEEEtran}

%%%%%%%%%%%%%%%%%%%%%%%%%%%%%%%%%%%%%%%%%%%%%%%%%%%%%%%%%%%%%%%%%%%%%%%%%%%%%%%


% =============================================================================
% Include packages
% =============================================================================

\usepackage[cmex10]{amsmath}  % needed e.g. for align-environment
\usepackage{amssymb}  % needed e.g. for \lesssim
\usepackage{cite}
\usepackage[capitalise]{cleveref}  % auto-prepend Eq., Fig., Tab., etc.
\usepackage[T1]{fontenc}
\usepackage{graphicx}

\usepackage[utf8x]{inputenc}
\usepackage{microtype}  % improves readability and saves space
\usepackage{siunitx}  % for correct typesetting of units, e.g. ohm
%\usepackage{subcaption}
\usepackage{upgreek}  % needed for bold 'micro' sign in abstract
\usepackage{xspace}  % needed for bold 'micro' sign in abstract

\usepackage{tikz}

% =============================================================================
% Own macros
% =============================================================================

% Define dBm as \dbm and re-define dB as \db. This is against the rules of
% siunitx, which prefers defining dB as \deci\bel (hence, dBm would be defined
% as \deci\belm.

\DeclareSIUnit\db{dB}
\DeclareSIUnit\dbm{dBm}

\DeclareSIUnit\MHz{MHz}
\DeclareSIUnit\GHz{GHz}

\newcommand{\boldum}{\,\boldmath$\upmu$m\xspace}  % bold 'um' in abstract

\newcommand{\VDC}{V_\text{DC}}    % DC voltage
\newcommand{\IDC}{I_\text{DC}}    % DC current
\newcommand{\PDC}{P_\text{DC}}    % DC power
\newcommand{\pDC}{p_\text{DC}}    % DC power, normalized
\newcommand{\PRF}{P_\text{RF}}    % fundamental RF power
\newcommand{\pRF}{p_\text{RF}}    % fundamental RF power, normalized
\newcommand{\etaD}{\eta_\text{D}} % drain efficiency

\newcommand{\vB}{v_\text{B}}   % normalized class-B voltage
\newcommand{\Vk}{V_\text{k}}   % knee voltage
\newcommand{\vJ}{v_\text{J}}   % normalized reactive class-J voltage
\newcommand{\vRJ}{v_\text{RJ}} % normalized resistive-reactive class-J voltage

% =============================================================================
% Own setup
% =============================================================================

% setup the graphics path and their extensions so you won't have to specify these 	  with every instance of \includegraphics
\graphicspath{{graphics/}}
\DeclareGraphicsExtensions{.pdf,.jpeg,.png,.pdf_tex}

% font of si units adapted by the font of the environment
\sisetup{detect-all=true}

% glossary, abbreveation in the text
\usepackage[acronym]{glossaries}
	\setacronymstyle{short-long}
	
% =============================================================================
% Glossary
% =============================================================================

\newglossaryentry{rfdac}{name={RF DAC},description={radio-frequency digital-to-analog converter},firstplural={radio-frequency digital-to-analog converters (RF DACs)},first={radio-frequency digital-to-analog converter (RF DAC)}}

\newglossaryentry{snr}{name={SNR},description={signal to noise ratio},firstplural={signal-to-noise ratios (SNRs)},first={signal-to-noise ratio (SNR)}}

\newglossaryentry{hemt}{name={HEMT},description={high-electron-mobility transistor},firstplural={high-electron-mobility transistors (HEMTs)},first={high-electron-mobility transistor (HEMT)}}

\newglossaryentry{gan}{name={GaN},description={gallium nitride},first={gallium nitride (GaN)}}

\newglossaryentry{pa}{name={PA},description={power amplifier},firstplural={power amplifiers (PAs)},first={power amplifier (PA)}}

\newglossaryentry{mmic}{name={MMIC},description={monolithic microwave integrated circuit},firstplural={monolithic microwave integrated circuits (MMICs)},first={monolithic microwave integrated circuit (MMIC)}}

\newglossaryentry{pcm}{name={PCM},description={pulse-code modulation},first={pulse-code modulation (PCM)}}

\newglossaryentry{osr}{name={OSR},description={oversampling ratio},first={oversampling ratio (OSR)}}

\newglossaryentry{rffe}{name={RFFE},description={radio-frequency front-end},first={radio-frequency front-end (RFFE)}}

\newglossaryentry{dut}{name={DUT},description={device under test},first={device under test (DUT)}}

\newglossaryentry{awg}{name={AWG},description={arbitrary waveform generator},first={arbitrary waveform generator (AWG)}}
%%%%%%%%%%%%%%%%%%%%%%%%%%%%%%%%%%%%%%%%%%%%%%%%%%%%%%%%%%%%%%%%%%%%%%%%%%%%%%%


\begin{document}

\title{Design, Realization, and Evaluation of a Riemann Pump in GaN Technology}
\author{%
	Markus~Wei\ss{},
    Christian~Friesicke,
	Thomas Metzger,    
    Edgar Schmidhammer,
    R\"{u}diger~Quay,
    and Oliver~Ambacher 
%    Fraunhofer Institute for Applied Solid State Physics\\
%    Freiburg, Germany\\
%    markus.weiss@iaf-extern.fraunhofer.de%	markus.weiss@iaf-extern.fraunhofer.de
    %
    \thanks{%
      Manuscript received XXX YY, ZZZZ;
      revised XXX YY, ZZZZ;
      accepted XXX YY, ZZZZ.
      Date of publication Month XX, ZZZZ;
      date of current version XXX YY, ZZZZ.      
    }%    
    \thanks{M.~Wei\ss{}, C.~Friesicke, R.~Quay and O.~Ambacher are with the Fraunhofer Institute for Applied Solid-State
      Physics (IAF), 79108 Freiburg, Germany (E-mail: markus.weiss@iaf-extern.fraunhofer.de).%
    }%
    \thanks{
      T. Metzger and E. Schmidhammer are with the EPCOS AG, 81671 Munich, Germany.%
    }%
    \thanks{%
      Color versions of one or more of the figures in this letter are available
      online at http://ieeexplore.ieee.org.}
    \thanks{%
      Digital Object Identifier XXXXXXX%
    }%
}

\markboth{IEEE Microwave Wireless and Components Letters,%
          Vol. XX, No. X, Month 20YY}%
         {Wei\ss{} \MakeLowercase{\textit{et al.}}:
          A Riemann Pump in GaN Technology}
\maketitle


% =============================================================================
% Abstract
% =============================================================================
\begin{abstract}
The realisation of a novel Riemann Pump for \glspl{rfdac} in combination with a power amplifier is presented that improves the \gls{snr} of conventional converter concepts.
The presented concept results in an arbitrary waveform generator that at the same time is capable to provide several watts of output power at RF-frequencies to 50 Ohms.
Further, the \gls{gan} \gls{hemt} technology provides high-switching frequencies to ensure an oversampling ratio of 5 for a wide baseband bandwidth. 
The Riemann Pump, which is controlled with a digital bit-stream, is based on the current-steering topology and provides the possibility to synthesize arbitrary waveforms.
A two-bit \gls{rfdac} was designed with multiple GaN \glspl{mmic} and proves the feasibility to generate arbitrary waveforms.
Measurement results yield triangular signals with a baseband frequency of 100 MHz for an input-control data rate of 200 Mbps.
\end{abstract}
%
\begin{IEEEkeywords}
  AWG, DAC, transmitter architecture.
\end{IEEEkeywords}
%
% =============================================================================
% Introduction
% =============================================================================
%
\section{Introduction}
\label{sec:introduction}
\IEEEPARstart{T}{he} immense demand for high-data rates in mobile communication leads researchers to investigate new architectures to improve the conventional concepts
for data conversion and power amplification.
One concept, that improves the performance of the DAC-PA chain, is the Riemann Pump \cite{VeyracRivetDevalEtAl2014}, \cite{DevalRivetVeyrac2015}.
The aim to develop a Riemann Pump in \gls{gan} technology is to generate a high power modulated microwave signal that is suitable for the next generation of mobile communication.
The possibility to cover a frequency range from DC to 6 GHz enables this architecture to operate with all common mobile communication standards.
The advantage of \gls{gan} is to switch high power at high frequencies as compared to conventional CMOS switches, which are limited in power.
As the power consumption of \gls{gan} \glspl{hemt} typically exceeds the consumption limit for mobile devices, the presented architecture is suitable, e.g., for base station transmitters.\\
The presented concept of the Riemann Pump improves the \gls{snr} compared to conventional \gls{pcm} converters.
\\
This paper presents the first realization of a Riemann Pump in \gls{gan} technology.
The new concept of the Riemann Pump shows the potential to improve the data rate for base station transmitters.
A circuit is devised and designed in Chapter \ref{sec:theory} based on the presented idea.
Chapter \ref{sec:assembly} covers the implementation and assembly of a first Riemann Pump demonstrator.
In Chapter \ref{sec:experiment} measurement results verify the feasibility of the concept by the generation of a triangular signal, which is representative for an arbitrary signal waveform.
%
% =============================================================================
% Concept
% =============================================================================
%
\section{Concept of the Riemann Pump circuit}
\label{sec:theory}

In this section a \gls{rfdac} is described which is derived from the concept of a charge pump.
The digital-to-analog conversion is based on the current steering topology and pumps charges into a capacitive output load.
The current into the capacitive load is integrated over time to form the resulting voltage.
Hence, this custom charge pump is named after the inventor of the Riemann integral, Bernhard Riemann.

A \gls{pcm} converter improves the \gls{snr} by 5 to 7 dB for every one-bit increase of resolution, while the Riemann conversion improves it by 5 to 10 dB \cite{VeyracRivetDevalEtAl2016}.
Every doubling of the \gls{osr}, which is defined as $\text{OSR} = \frac{f_{\text{sampling}}} {2 f_{\text{signal,max}}}$, improves the \gls{snr} of a \gls{pcm} converter by 3 dB, while the Riemann conversion improves it by 9 dB.
The mathematical approximation for the \gls{snr} of the Riemann Pump is given in Equation \ref{eq:snr_riemannconversion}.
For further details and the derivation see \cite{VeyracRivetDevalEtAl2016}.
%
	\begin{align}
  \textit{SNR}_{\text{dB}} \approx 6.02N + 9.03r - 7.78 + 10 \log \left(1- \frac{1}{2^{N-1}} + \frac{1}{2^{2N}}\right)
    \label{eq:snr_riemannconversion}
	\end{align}
%
The factor $N$ represents the number of bits used for the resolution, while $r$ is the binary logarithm of the \gls{osr}.
% -----------------------------------------------------------------------------
\begin{figure}[htb]
  \centering
	\begin{scriptsize}
  	\def\svgwidth{\columnwidth}
 	\input{graphics/schematic_RP_multibit_smallScale.pdf_tex} 
  	\caption{Schematic of the Riemann Pump. Marked on the left side is the driver circuit. The push-pull stages are connected in parallel.}
  	\label{fig:schematic_multibit_rp}
	\end{scriptsize}
\end{figure}
% -----------------------------------------------------------------------------
%
Figure \ref{fig:schematic_multibit_rp} shows the schematic of the Riemann Pump, where the push-pull stages are cascaded in parallel.
The current-drive capability of the power transistors in parallel is increased linearly with the power of 2 to ensure a correct encoding.
The digital driver circuit is marked with dashed lines and is necessary for each single stage. % cascaded in parallel.
It enables to control the power transistors, which are intended to be voltage-controlled current sources, with a digital bit stream.
A co-integrated power amplifier, serving as the output load, enables to reduce the number of external components in the \gls{rffe}.
The capacitive input impedance of the power amplifier makes it suitable for this concept.
Based on a charge pump the resulting output voltage at the load is defined as:
%
\begin{align}
  V_{\text{out}} =  \frac{1}{C_{\text{out}}} \int_0^t \! i_{\text{out}}(\tau) \, \mathrm{d}\tau.
    \label{eq:output_voltage_integral}
\end{align}
%
Equation \ref{eq:output_voltage_integral} states that the voltage is proportional to the integral of the current to the load.
This technique makes it possible to synthesize arbitrary RF-voltage waveforms at the \gls{pa} by varying the current.
Absolutely essential to convert digital input data into a defined analog output signal is to establish a defined set of current amplitudes which charge and discharge the output capacitance, here the input impedance of the \gls{pa}.
For a good approximation of the synthesized signal, a high sampling frequency, as well as stable current sources are needed.
In Figure \ref{fig:SlopeSynthSignal} (a) eight different slopes represent the change of the output voltage for a given sampling interval.
These eight slopes correspond to a three-bit resolution of the DAC. 
Figure \ref{fig:SlopeSynthSignal} (b) illustrates an example of a synthesized output signal using these slopes.
%
% -----------------------------------------------------------------------------
\begin{figure}[htb]
  \centering
	\begin{scriptsize}
  	\def\svgwidth{\columnwidth}
 	\input{graphics/SlopeSynthSignal.pdf_tex} 
  	\caption{(a) Representation of relative slopes and (b) signal generation with riemann code.}
  	\label{fig:SlopeSynthSignal}
	\end{scriptsize}
\end{figure}
% -----------------------------------------------------------------------------
%
The solid black line represents the desired output signal to be synthesized by the Riemann Pump.
For each sampling point the slope that minimizes the error between the sampled desired and the synthesized signal is chosen, which is illustrated by the dashed line.
As eight different slopes are generated, it corresponds with a three-bit resolution of the \gls{rfdac} enabling an encoding of each slope with a digital bit stream.
It is possible to control the output signal with a digital input stream representing the sequence of slopes to synthesize the signal.
In order to generate eight different currents the concept of a charge pump in a push-pull configuration is used.
The high-side transistors contribute to an increase of the output signal while the low-side transistors decrease the amplitude of the output signal.
Mandatory for the correct functioning of the push-pull configuration is a digital driver circuit to ensure a proper switching of the high-side transistors.
To prevent a signal delay caused by this driver circuit, this driver is also used for the low-side transistor.
%
% =============================================================================
% Implementation
% =============================================================================
%
% -----------------------------------------------------------------------------
\section{Implementation and assembly of a demonstrator}
\label{sec:assembly}
% -----------------------------------------------------------------------------
To the best of the author's knowledge, the first working demonstrator in \gls{gan} technology is presented.
To ensure a feasible measurement setup, the \gls{rfdac} is realized with a two-bit resolution, which is sufficient to prove the concept.
For the digital switching of the \gls{gan} power transistors, a switch-mode \gls{mmic} is used that has an implemented driver circuit \cite{MaroldtBruecknerQuayEtAl2014}.
These \glspl{mmic} are designed and fabricated in the \SI{0.25}{\micro\meter} AlGaN/GaN HEMT technology \cite{MaroldtDriverConcept}.
Figure \ref{fig:assembled_demonstrator} (a) illustrates the schematic, where the grey painted areas represent a single \gls{mmic}.
In Figure \ref{fig:assembled_demonstrator} (b) the layout of the realized demonstrator is shown.
%
% -----------------------------------------------------------------------------
\begin{figure}[htb]
  \centering
	\begin{scriptsize}
  	\def\svgwidth{\columnwidth}
 	\input{graphics/circuit_schematic_layout_ddrixy6.pdf_tex} 
  	\caption{(a) Schematic of assembly; the used \glspl{mmic} are highlighted in grey. (b) Assembled demonstrator layout.}
  	\label{fig:assembled_demonstrator}
	\end{scriptsize}
\end{figure}
% -----------------------------------------------------------------------------
%
The green rectangles represent off-chip bypass capacitors, the black rectangles are the \glspl{mmic} used.
Each \gls{mmic}, consisting of the grey highlighted circuits in Figure \ref{fig:assembled_demonstrator} (a), have got a via to the backside 
at the power transistors source potential.
Due to the microstrip approach for the \gls{mmic} it is necessary to isolate the \glspl{mmic} for the high-side transistors Q1 and Q3.
This isolation is realized by an isolated pad on the substrate, which induces an additional contribution to the thermal conductivity.
In order to reduce the impact of phase delays of the signal, the input and output lines as well as the bond wires are chosen for the same length.
\\
The full Riemann Pump is shown in Figure \ref{fig:photo_chipconnection_demonstrator}.
A close-up of assembled \glspl{mmic} is shown in Figure \ref{fig:photo_chipconnection_demonstrator} (a) according to the layout in Figure \ref{fig:assembled_demonstrator} (b).
This close-up is highlighted in the photography of the demonstrator in Figure \ref{fig:photo_chipconnection_demonstrator} (b).
% -----------------------------------------------------------------------------
\begin{figure}[htb]
  \centering
	\begin{scriptsize}
  	\def\svgwidth{\columnwidth}
 	\input{graphics/PhotoChipConnect_Demonstrator_compressed.pdf_tex} 
  	\caption{(a) Close-up of assembled \glspl{mmic}, (b) realized demonstrator.}
  	\label{fig:photo_chipconnection_demonstrator}
	\end{scriptsize}
\end{figure}
% -----------------------------------------------------------------------------
The demonstrator is of the size 50x60 $\text{mm}^2$, has four digital input and one analog output line, and DC supply connectors with decoupling network.
The analog output is at the top middle while the other four signal paths are the digital inputs.
%
% =============================================================================
% Measurement
% =============================================================================
%
% ------------------------------------------------------------------------------
\section{Time-domain measurement of synthesized output signal at RF- frequencies}
\label{sec:experiment}
% ------------------------------------------------------------------------------
To prove that the built demonstrator can convert digital input streams into an analog output signal, a time-domain measurement is performed.
A custom control and measurement setup is applied to ensure excitation of the \gls{dut} with correct amplitude and phase.
Four square-wave signals are applied by an \gls{awg} (Keysight M8195A) to represent the digital data stream. 
These signals have to be amplified by a broadband pre-amplifier and shifted in the DC offset with bias tees to ensure proper switching of the transistors Qb1, Qb2, Qb3 and Qb4 at the input. A stability analysis, where the biasing is checked to stay constant, is performed to ensure that the \gls{dut} does not oscillate.
Further, both high-side transistors (Q1, Q3) are controlled with an in-phase signal illustrated in Figure \ref{fig:meas_Input_Output_RLoad_100M_SmallSize_Paper} (a) (grey), while the inverse signal (black) is applied to both low-side transistors (Q2, Q4).
This differential digital input stream is generated by the \gls{awg} for a data rate of up to 200 Mbps in the time-domain.
The \gls{dut} is loaded with a $50 \Omega$ resistor and is controlled by this input signal, which leads to the output signal shown in Figure \ref{fig:meas_Input_Output_RLoad_100M_SmallSize_Paper} (b).
%
% -----------------------------------------------------------------------------
\begin{figure}[htb]
  \centering
	\begin{scriptsize}
  	\def\svgwidth{\columnwidth}
 	\input{graphics/meas_Input_Output_RLoad_100M_SmallSize_Paper.pdf_tex} 
  	\caption{(a) Differential input control signal and (b) corresponding output signal; theoretical signal (dashed), measured signal (solid).}
  	\label{fig:meas_Input_Output_RLoad_100M_SmallSize_Paper}
	\end{scriptsize}
\end{figure}
% -----------------------------------------------------------------------------
%
As expected for a push-pull measurement the output signal switches between Vdd (5V) and GND (0V).
The dashed line represents the output of an ideal switch while the solid line shows the measured output signal of the \gls{dut}.
To show the feasibility to synthesize different output signals it is necessary to generate different currents, which are integrated by a capacitor, as mentioned in Chapter \ref{sec:theory}.
Therefore the resistive load is replaced by a capacitive load, representing the capacitive input impedance characteristic of a \gls{pa}.
In-phase switching of both high-side and low-side transistors led to the results shown in Figure \ref{fig:meas_Output_CLoad_100M_1io_3io} (b).
Here the capacitor is charged with the maximum available current as both high-side transistors simultaneously drive current to the load, hence the biggest slope is chosen.
Encoding the biggest slope with the bit-sequence "00" the corresponding most negative slope is "11" while "01" and "10" contribute to the smaller slopes in between.
In order to select the slope $1 i_0$ both high-side as well as both low-side transistors are driven with anti-phase signals.
The slope $1 i_0$ is illustrated in Figure \ref{fig:meas_Output_CLoad_100M_1io_3io} (a) while the slope $3 i_0$ is shown in (b).
%
% -----------------------------------------------------------------------------
\begin{figure}[htb]
  \centering
	\begin{scriptsize}
  	\def\svgwidth{\columnwidth}
 	\input{graphics/meas_Output_CLoad_100M_1i0_3i0.pdf_tex} 
  	\caption{ideal slope (dashed); measurement (solid). Slope of (a) $\pm 1  i_0$ and (b) $\pm 3 i_0$.}
  	\label{fig:meas_Output_CLoad_100M_1io_3io}
	\end{scriptsize}
\end{figure}
% -----------------------------------------------------------------------------
The notations of both figures are the same, the solid black line represents the measured time-domain signal for the frequency of 100 MHz, while the dashed line represents the ideal signal waveform.
The measurement yields an output power of 1.38 W for Vdd = 5V, which shows the possibility to increase the output power to several watts by applying voltages of up to 28V for Vdd.


% ------------------------------------------------------------------------------
\section{Conclusions}
\label{sec:conclusion}
% ------------------------------------------------------------------------------
A prototype of a Riemann Pump in \gls{gan} technology is presented to validate the concept of this current steering topology at RF-frequencies.
The measurement results show the feasibility of synthesizing arbitrary waveforms at 100 MHz.
The potential of the chosen technology promises to cover even higher frequencies, for \gls{mmic} integration 
of the concept, to satisfy the condition for the new mobile communication standards.

% ------------------------------------------------------------------------------
% Appendices
% ------------------------------------------------------------------------------

%\section*{Acknowledgement}
%The authors would like to thank Dirk Meder at Fraunhofer IAF for the assistance with the assembly process.

% ------------------------------------------------------------------------------
% References
% ------------------------------------------------------------------------------

\bibliographystyle{IEEEtran}
\bibliography{IEEEabrv,refs}


\end{document}

% EOF
