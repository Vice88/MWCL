%%%%%%%%%%%%%%%%%%%%%%%%%%%%%%%%%%%%%%%%%%%%%%%%%%%%%%%%%%%%%%%%%%%%%%%%%%%%%%%

\documentclass[journal]{IEEEtran}

%%%%%%%%%%%%%%%%%%%%%%%%%%%%%%%%%%%%%%%%%%%%%%%%%%%%%%%%%%%%%%%%%%%%%%%%%%%%%%%


% =============================================================================
% Include packages
% =============================================================================

\usepackage[cmex10]{amsmath}  % needed e.g. for align-environment
\usepackage{amssymb}  % needed e.g. for \lesssim
\usepackage{cite}
\usepackage[capitalise]{cleveref}  % auto-prepend Eq., Fig., Tab., etc.
\usepackage[T1]{fontenc}
\usepackage{graphicx}
\usepackage[utf8x]{inputenc}
\usepackage{microtype}  % improves readability and saves space
\usepackage{siunitx}  % for correct typesetting of units, e.g. ohm
%\usepackage{subcaption}
\usepackage{upgreek}  % needed for bold 'micro' sign in abstract
\usepackage{xspace}  % needed for bold 'micro' sign in abstract

\usepackage{tikz}

% =============================================================================
% Own macros
% =============================================================================

% Define dBm as \dbm and re-define dB as \db. This is against the rules of
% siunitx, which prefers defining dB as \deci\bel (hence, dBm would be defined
% as \deci\belm.

\DeclareSIUnit\db{dB}
\DeclareSIUnit\dbm{dBm}

\DeclareSIUnit\MHz{MHz}
\DeclareSIUnit\GHz{GHz}

\newcommand{\boldum}{\,\boldmath$\upmu$m\xspace}  % bold 'um' in abstract

\newcommand{\VDC}{V_\text{DC}}    % DC voltage
\newcommand{\IDC}{I_\text{DC}}    % DC current
\newcommand{\PDC}{P_\text{DC}}    % DC power
\newcommand{\pDC}{p_\text{DC}}    % DC power, normalized
\newcommand{\PRF}{P_\text{RF}}    % fundamental RF power
\newcommand{\pRF}{p_\text{RF}}    % fundamental RF power, normalized
\newcommand{\etaD}{\eta_\text{D}} % drain efficiency

\newcommand{\vB}{v_\text{B}}   % normalized class-B voltage
\newcommand{\Vk}{V_\text{k}}   % knee voltage
\newcommand{\vJ}{v_\text{J}}   % normalized reactive class-J voltage
\newcommand{\vRJ}{v_\text{RJ}} % normalized resistive-reactive class-J voltage

% =============================================================================
% Own setup
% =============================================================================

% setup the graphics path and their extensions so you won't have to specify these 	  with every instance of \includegraphics
\graphicspath{{graphics/}}
\DeclareGraphicsExtensions{.pdf,.jpeg,.png,.pdf_tex}

% font of si units adapted by the font of the environment
\sisetup{detect-all=true}

% glossary, abbreveation in the text
\usepackage[acronym]{glossaries}
	\setacronymstyle{short-long}
	
% =============================================================================
% Glossary
% =============================================================================

\newglossaryentry{rfdac}{name={RF DAC},description={radio-frequency digital-to-analog converter},firstplural={radio-frequency digital-to-analog converters (RF DACs)},first={radio-frequency digital-to-analog converter (RF DAC)}}

\newglossaryentry{snr}{name={SNR},description={signal to noise ratio},firstplural={signal-to-noise ratios (SNRs)},first={signal-to-noise ratio (SNR)}}

\newglossaryentry{hemt}{name={HEMT},description={high electron mobility transistor},firstplural={high electron mobility transistors (HEMTs)},first={high electron mobility transistor (HEMT)}}

\newglossaryentry{gan}{name={GaN},description={gallium nitride},first={gallium nitride (GaN)}}

\newglossaryentry{pa}{name={PA},description={power amplifier},firstplural={power amplifiers (PAs)},first={power amplifier (PA)}}

\newglossaryentry{mmic}{name={MMIC},description={monolithic microwave integrated circuit},firstplural={monolithic microwave integrated circuits (MMICs)},first={monolithic microwave integrated circuit (MMIC)}}

\newglossaryentry{pcm}{name={PCM},description={pulse-code modulation},first={pulse-code modulation (PCM)}}

\newglossaryentry{osr}{name={OSR},description={oversampling ratio},first={oversampling ratio (OSR)}}

\newglossaryentry{rffe}{name={RFFE},description={radio-frequency front-end},first={radio-frequency front-end (RFFE)}}

\newglossaryentry{dut}{name={DUT},description={device under test},first={device under test (DUT)}}

\newglossaryentry{awg}{name={AWG},description={arbitrary waveform generator},first={arbitrary waveform generator (AWG)}}
%%%%%%%%%%%%%%%%%%%%%%%%%%%%%%%%%%%%%%%%%%%%%%%%%%%%%%%%%%%%%%%%%%%%%%%%%%%%%%%


\begin{document}

\title{Investigation, Design, and Evaluation of a Riemann Pump in GaN Technology}
% if 5G is mentioned, it should be explained.
\author{%
	Markus~Wei\ss{},
    Christian~Friesicke,
    R\"{u}diger~Quay,
    and~Oliver~Ambacher\\
    Fraunhofer Institute for Applied Solid State Physics\\
    Freiburg, Germany\\
    markus.weiss@iaf-extern.fraunhofer.de%	markus.weiss@iaf-extern.fraunhofer.de
    %
    \thanks{%
      Manuscript received XXX YY, ZZZZ;
      revised XXX YY, ZZZZ;
      accepted XXX YY, ZZZZ.
      Date of publication Month XX, ZZZZ;
      date of current version XXX YY, ZZZZ.      
    }%    
    \thanks{M.~Wei\ss{}, C.~Friesicke, R.~Quay and O.~Ambacher are with the Fraunhofer Institute for Applied Solid-State
      Physics (IAF), 79108 Freiburg, Germany.%
    }%
    \thanks{%
      Color versions of one or more of the figures in this letter are available
      online at http://ieeexplore.ieee.org.}
    \thanks{%
      Digital Object Identifier XXXXXXX%
    }%
}

\markboth{IEEE Microwave Wireless and Components Letters,%
          Vol. XX, No. X, Month 20YY}%
         {Wei\ss{} \MakeLowercase{\textit{et al.}}:
          A Riemann Pump in GaN Technology}
\maketitle


% =============================================================================
% Abstract
% =============================================================================

\begin{abstract}
%
A novel architecture for \glspl{rfdac} is investigated that improves the \gls{snr} of conventional converter concepts.
The presented concept results in an arbitrary waveform generator that is capable to provide several watts of output power.
Further, the \gls{hemt} technology provides high switching frequencies to ensure an oversampling ratio of 5 for a wide baseband bandwidth. %signal.
%to cover a wide baseband signal.
% to cover a baseband frequency from DC to 6 GHz while ensuring an oversampling ratio of the signal of 5.
The utilization of \gls{gan} enables to design a one-chip solution of a \gls{rfdac} co-integrated with a \gls{pa}, which is named Riemann Pump.
%An integrated digital driver circuit makes it possible to process a highly integrated one chip solution for the DAC co-integrated with a power amplifier in GaN (gallium nitride) technology.
%The presented demonstrator has yielded baseband frequencies in the range of several hundred MHz.
The Riemann Pump, which is controlled with a digital bit-stream, is based on the current steering topology and provides the possibility to synthesize arbitrary waveforms.
A two-bit \gls{rfdac} designed with multiple \glspl{mmic} proofs the feasibility to generate arbitrary waveforms.
Measurement results yield triangular signals with a baseband frequency of 100 MHz for an input-control data rate of 200 Mbps.

%brandnew concept and integration of a push pull stage for high power applications.
%novel circuit design regarding concept.
%intelligent assembly for signal lines, also thermal transfer is improved considered.
%very clever measurement strategies.
%since input control strategy is not the easiest, input buffer not available.
%signal generator + preamp + bias tee (wideband since digital signals).
%output measurement results - different signal synthesis.
%DIGITAL HIGH SPEED DRIVER FOR GAN POWER APPLICATION.
%FIRST EVER BUILT GAN DEMONSTRATOR FOR THE CONCEPT OF RP.\\%%
%
\end{abstract}
%
\begin{IEEEkeywords}
  AWG, DAC, transmitter architecture.
\end{IEEEkeywords}
%
% =============================================================================
% Introduction
% =============================================================================
%
\section{Introduction}
\label{sec:introduction}
\IEEEPARstart{T}{he} immense demand for high data rates in mobile communication leads researchers to investigate new architectures to improve the conventional concepts.
One concept that improves the performance is the Riemann Pump \cite{VeyracRivetDevalEtAl2014}.
The aim to develop a Riemann Pump in \gls{gan} technology is to generate a high power modulated microwave signal, which is suitable for the next generation of mobile communication.
The possibility to cover a frequency range from DC to 6 GHz enables this architecture to operate with all common mobile communication standards.
%Due to the fact that the sum of several signals of the common mobile communication standard can be generated, a concurrent transmitter is built which will increases the data rate.
%The possibility to generate any modulated RF signal leads to a much higher data rate as with conventional concepts of RF-FE.
%This idea is based on software defined radios.
%In fact of the summation of different signals enables this technique to concurrent transmit different signals and hence increase the data rate.
The advantage of \gls{gan} is to switch high power at high frequencies as conventional CMOS switches are limited in power.
As the power consumption of \gls{gan} \glspl{hemt} exceeds the limits for mobile devices the presented architecture is suitable for base station transmitters.\\
The presented concept of the Riemann Pump improves the \gls{snr} compared to conventional \gls{pcm} converters.
%Common concepts and hardware topology as CMOS are limited in switching frequency or in the power.
%The technology of high electron mobility transistors in GaN enables a switching frequency of 30 GHz which can synthesize signals of 6 GHz baseband frequency with an oversampling of 5.
%Synthesizing arbitrary signals with a bandwidth of 5 GHz enables to cover the common mobile communication standards.
%This is the idea of software defined radio, where it is to bring the digital domain as close as possible to the analog.
A \gls{pcm} converter improves the \gls{snr} by 5 to 7 dB for every one-bit increase of resolution while the Riemann conversion improves it by 5 to 10 dB \cite{VeyracRivetDevalEtAl2016}.
Every doubling of the \gls{osr}, which is defined as $\text{OSR} = \frac{f_{\text{sampling}}} {2 f_{\text{signal,max}}}$, improves the \gls{snr} of a \gls{pcm} converter by 3 dB while the Riemann conversion improves it by 9 dB.
The mathematical approximation for the \gls{snr} of the Riemann Pump is given in Equation \ref{eq:snr_riemannconversion}.
For further details and the derivation see \cite{VeyracRivetDevalEtAl2016}.
%
\begin{align}
  \textit{SNR}_{\text{dB}} \approx 6.02N + 9.03r - 7.78 + 10 \log \left(1- \frac{1}{2^{N-1}} + \frac{1}{2^{2N}}\right)
    \label{eq:snr_riemannconversion}
\end{align}
%
The factor $N$ represents the number of bits used for the resolution, while $r$ is the binary logarithm of the \gls{osr}.
%is the exponent of the \gls{osr} which is defined as  = 2^r.
% $SNR = 6,02N + 3,01r + 1,76$ while the Riemann conversion yields $SNR = 6,02N + 9,03r - 7,78 + 10log(1- \frac{1}{2^{N-1}} + \frac{1}{2^{2N}})$ \cite{VeyracRivetDevalEtAl2016}.
\\
This paper presents the first realization of a Riemann Pump in \gls{gan} technology.
The new concept of the Riemann Pump shows the feasibility to improve the data rate.
A circuit is devised and designed in Chapter \ref{sec:theory} based on the presented idea.
Chapter \ref{sec:assembly} covers the implementation and assembly of a first Riemann Pump demonstrator.
In Chapter \ref{sec:experiment} measurement results verify the feasibility of the concept by the generation of a triangular signal, which is representative for any signal waveform.
Chapter \ref{sec:conclusion} summarizes the paper and give a short outlook for future work.
%A challenging task was to ensure the proper switching of a push-pull stage only consisting of depletion-mode n-doped transistors.
%A proper digital driver circuit is investigated to allow a proper switching of the HEMT which source potential is not fixed to ground potential rather changing the potential.
%This realized push-pull configuration in GaN technology is the first ever realized, to the best of the authors knowledge.
%The Riemann Pump is able to generate arbitrary signal waveforms to synthesize any modulated rf signal.
%With the co-integration of the digital-to-analog converter with a power amplifier in GaN technology, further external RF front end components are not necessary.
%As the GaN HEMTs are able to switch with a frequency of 30 GHz this results in a synthesized baseband signal frequency of 6 GHz while ensuring an oversampling of 5.


%
%
\section{Concept of the Riemann Pump circuit}
\label{sec:theory}
%
In this section a \gls{rfdac} is described which is derived from the concept of a charge pump.
The digital-to-analog conversion is based on the current steering topology and pumps charges into a capacitive output load.
The current into the capacitive load is integrated over time to form the resulting voltage.
Hence, this custom charge pump is named after the inventor of the Riemann integral, Bernhard Riemann.
%
% -----------------------------------------------------------------------------
\begin{figure}[htb]
  \centering
	\begin{scriptsize}
  	\def\svgwidth{\columnwidth}
 	\input{graphics/schematic_RP_multibit_smallScale.pdf_tex} 
  	\caption{Schematic of the Riemann Pump. Marked on the left side is the driver circuit. The push-pull stages are connected in parallel.}
  	\label{fig:schematic_multibit_rp}
	\end{scriptsize}
\end{figure}
% -----------------------------------------------------------------------------
%
Figure \ref{fig:schematic_multibit_rp} shows the schematic of the Riemann Pump, where the single stages are cascaded in parallel.
The current-drive capability of the power transistors in parallel is increased linearly with the power of 2 to ensure the correct encoding.
The digital driver circuit is marked with dashed lines and is necessary for each single stage cascaded in parallel.
An implemented power amplifier, serving as the output load, enables to reduce the number of external components in the \gls{rffe}.
The capacitive input impedance of the power amplifier makes it suitable for this concept.
The driver circuit enables to control the power transistors, which are intended to be a voltage-controlled current sources, with a digital bit stream.
Based on a charge pump the resulting output voltage at the load is defined as:
%
\begin{align}
  V_{\text{out}} =  \frac{1}{C_{\text{out}}} \int_0^t \! i_{\text{out}}(\tau) \, \mathrm{d}\tau.
    \label{eq:output_voltage_integral}
\end{align}
%
Equation \ref{eq:output_voltage_integral} states the dependence of the voltage through the current and time at the output load.
This technique makes it possible to synthesize arbitrary voltage waveforms at the \gls{pa} by varying the current.
Absolutely essential to convert digital input data into a defined analog output signal is to establish a defined set of current amplitudes which charge and discharge the output capacitance, here the input impedance of the \gls{pa}.
%generate different output signals is to establish different current amplitudes to charge and discharge the output capacitor.
%To generate different output signals it is necessary to establish different constant current amplitudes to charge and discharge the capacitor.
For a good approximation of the synthesized signal, a high sampling frequency as well as stable current sources are needed.
%Consistent current sources as well as a high sampling frequency were needed to ensure a high signal integrity.
In Fig. \ref{fig:SlopeSynthSignal}(a) eight different slopes represent the change of the output voltage for a given sampling interval.
% four different current amplitudes plus their direction times the sampling interval.
These eight slopes correspond to a three-bit resolution of the DAC. 
%To explain the concept in a proper way, eight different currents are established which corresponds to a three bit digital-to-analog converter.
Figure \ref{fig:SlopeSynthSignal}(b) illustrates an example of a synthesized output signal using these slopes.
%
% -----------------------------------------------------------------------------
\begin{figure}[htb]
  \centering
	\begin{scriptsize}
  	\def\svgwidth{\columnwidth}
 	\input{graphics/SlopeSynthSignal.pdf_tex} 
  	\caption{(a) Representation of relative slopes and (b) signal generation with riemann code.}
  	\label{fig:SlopeSynthSignal}
	\end{scriptsize}
\end{figure}
% -----------------------------------------------------------------------------
%
%This technology is capable to provide high power at high switching frequencies, which is intended to get a high sampling rate and hence a high signal integrity.
%To increase the signal integrity it was also possible to increase the sampling frequency and hence decrease sampling time to synthesize the desired signal.
The solid black line represents the desired output signal to be synthesized by the Riemann Pump.
For each sampling point, the slope that minimizes the error between the sampled desired and the synthesized signal is chosen.
As eight different slopes are generated, it corresponds with a three-bit resolution of the \gls{rfdac} enabling an encoding of each slope with a digital bit stream.
It is possible to control the output signal with a digital input stream representing the sequence of slopes to synthesize the signal.
In order to generate eight different currents the concept of a charge pump in a push-pull configuration is used.
The high-side transistors contribute to an increase of the output signal while the low-side transistors decrease the amplitude of the output signal.
Mandatory for the correct functioning of the push-pull configuration is a digital driver circuit to ensure a proper switching of the high-side transistors.
To prevent a signal delay caused by this driver circuit, this driver is also used for the low-side transistor.
%A huge advantage of this technology is, that it is able to provide high power while having an immense switching speed and hence synthesize a signal with high integrity.
%In fact of this the load of a the push-pull stage can be implemented as a power amplifier which is directly connected to the antenna for wave propagation.
%Therefore much external components are not necessary.
%Since the input impedance characteristic of a power transistor is capacitive this is utilized to generate the output signal.
%The load of the described concept consists of the input impedance of a power amplifier which amplifies the signal and propagate it to the antenna.
%The charging and discharging of the capacitive input impedance of the power amplifier led to the possibility to generate arbitrary waveforms.
%The circuit used for this simulation is based on the assembled device under test, to ensure proper comparison of the results.
%The simulation confirms the possibility to generate different signals using this concept.
%
% -----------------------------------------------------------------------------
\section{Implementation and assembly of a demonstrator}
\label{sec:assembly}
% -----------------------------------------------------------------------------
To the best of the author's knowledge, the first ever built demonstrator in \gls{gan} technology is presented.
To ensure a feasible measurement setup, the \gls{rfdac} is realized with a two-bit resolution which is sufficient to prove the concept.
%In order of limited measurement equipment the demonstrator is built with four inputs which corresponds to a two bit resolution.
%The first ever built demonstrator to proof the concept of the Riemann Pump is presented in the following subsection.
For the digital switching of the \gls{gan} power transistors a \gls{mmic} is used which has already implemented a proper driver circuit.
These \glspl{mmic} are designed and fabricated in the \SI{0.25}{\micro\meter} AlGaN/GaN HEMT technology by Fraunhofer IAF \cite{MaroldtDriverConcept}.
Figure \ref{fig:assembled_demonstrator} (a) illustrates the schematic where the grey painted areas represent a single \gls{mmic}.
In Fig. \ref{fig:assembled_demonstrator}(b) the layout of the realized demonstrator is shown.
%
% -----------------------------------------------------------------------------
\begin{figure}[htb]
  \centering
	\begin{scriptsize}
  	\def\svgwidth{\columnwidth}
 	\input{graphics/circuit_schematic_layout_ddrixy6.pdf_tex} 
  	\caption{(a) Schematic of assembly; the used \glspl{mmic} are highlighted in grey. (b) Assembled demonstrator layout.}
  	\label{fig:assembled_demonstrator}
	\end{scriptsize}
\end{figure}
% -----------------------------------------------------------------------------
%
The green rectangles represent off-chip bypass capacitors, and the black rectangles are the used \glspl{mmic}.
Each \gls{mmic}, consisting of the grey highlighted circuit in Figure \ref{fig:assembled_demonstrator} (a), have got a via to the backside at the power transistors source potential.
In fact of this via and the backside metallization it is necessary to isolate the \glspl{mmic} for the high-side transistors Q1 and Q3.
This isolation is realized by an isolated pad on the substrate, which induces a critical heat transfer.
In order to reduce the impact of phase delays of the signal the input and output lines as well as the bond wires are of the same length.
\\
%In fact of this the heat transfer is not optimal since the transistors generate several watts.
%Also two different layouts are designed to compare the property of heat transfer.
% Each MMIC used in this work consists of the elements shown in Fig \ref{fig:schematic_multibit_rp} for the low side, which include the driver circuit of two transistors and the resistor plus the output power transistor.
%In fact of the assembly, one version used a isolated pad in fact of the undesired backside metallisation of the chip.
%This is undesired since the output of the high side stage is the source pin.
%For the low side the source pin is grounded and the output is the drain of the power amplifier.
The first Riemann Pump in GaN technology is shown in Figure \ref{fig:photo_chipconnection_demonstrator}.
A close-up of assembled \glspl{mmic} is shown in Figure \ref{fig:photo_chipconnection_demonstrator} (a) according to the layout in Figure \ref{fig:assembled_demonstrator}(b).
This close-up is highlighted in the photography of the demonstrator in Figure \ref{fig:photo_chipconnection_demonstrator} (b).
% -----------------------------------------------------------------------------
\begin{figure}[htb]
  \centering
	\begin{scriptsize}
  	\def\svgwidth{\columnwidth}
 	\input{graphics/PhotoChipConnect_Demonstrator.pdf_tex} 
  	\caption{(a) Close-up of assembled \glspl{mmic}, (b) realized demonstrator.}
  	\label{fig:photo_chipconnection_demonstrator}
	\end{scriptsize}
\end{figure}
% -----------------------------------------------------------------------------
The demonstrator is of the size 50x60 $\text{mm}^2$, has four digital input and one analog output line and DC supply connectors with decoupling network.
The analog output is at the top middle while the other four signal paths are the digital inputs.
%
% ------------------------------------------------------------------------------
\section{Time-domain measurement of synthesized output signal}
\label{sec:experiment}
% ------------------------------------------------------------------------------

%The theoretical waveforms are experimentally verified using the demonstrator which is assembled with four MMICs as shown in Figure \ref{fig:assembled_demonstrator} (b). 
To prove that the built demonstrator can convert digital input streams into an analog output signal, a time-domain measurement is performed.
A custom control and measurement setup was applied to ensure excitation of the \gls{dut} with correct amplitude and phase.
Four input signals are applied by an \gls{awg} (Keysight M8195A) to represent the digital data stream. 
These signals had to be amplified by a broadband pre-amplifier and shifted in the DC offset with bias tees to ensure proper switching of the transistors Qb1, Qb2, Qb3 and Qb4 at the input.
First of all, a stability analysis is performed to ensure that the \gls{dut} does not oscillate.
Further, both high-side transistors (Q1, Q3) are controlled with an in-phase signal illustrated in Figure \ref{fig:meas_Input_Output_RLoad_100M_SmallSize_Paper}(a) (grey) , while the inverse signal (black) is applied to both low-side transistors (Q2, Q4).
This differential digital input stream is generated by the \gls{awg} for a data rate of 200 Mbps in the time-domain.
The \gls{dut} is loaded with a $50 \Omega$ resistor and is controlled by this input signal, which leads to the output signal shown in Figure \ref{fig:meas_Input_Output_RLoad_100M_SmallSize_Paper} (b).
%
% -----------------------------------------------------------------------------
\begin{figure}[htb]
  \centering
	\begin{scriptsize}
  	\def\svgwidth{\columnwidth}
 	\input{graphics/meas_Input_Output_RLoad_100M_SmallSize_Paper.pdf_tex} 
  	\caption{(a) Differential input control signal and (b) corresponding output signal; theoretical signal (dashed), measured signal (solid).}
  	\label{fig:meas_Input_Output_RLoad_100M_SmallSize_Paper}
	\end{scriptsize}
\end{figure}
% -----------------------------------------------------------------------------
%
As expected for a push-pull measurement the output signal switches between Vdd (5V) and GND (0V).
The dashed line represents the output of an ideal switch while the solid line shows the measured output signal of the \gls{dut}.
To show the feasibility to synthesize different output signals it is necessary to generate different currents which are integrated at a capacitor, as mentioned in Chapter \ref{sec:theory}.
Therefore the resistive load is replaced by a capacitive load, representing the capacitive input impedance characteristic of a \gls{pa}.
In-phase switching of both high-side and low-side transistors led to the results shown in Figure \ref{fig:meas_Output_CLoad_100M_1io_3io} (b).
Here the capacitor is charged with the maximum available current as both high-side transistors simultaneously drive current to the load, hence the biggest slope is chosen.
Encoding the biggest slope with the bit-sequence "00" the corresponding most negative slope is "11" while "01" and "10" contribute to the smaller slopes in between.
In order to select the slope $1 i_0$ both high-side as well as both low-side transistors are driven with anti-phase signals.
The slope $1 i_0$ is illustrated in Figure \ref{fig:meas_Output_CLoad_100M_1io_3io}(a) while the bigger slope $3 i_0$ is shown in (b).
%
% -----------------------------------------------------------------------------
\begin{figure}[htb]
  \centering
	\begin{scriptsize}
  	\def\svgwidth{\columnwidth}
 	\input{graphics/meas_Output_CLoad_100M_1i0_3i0.pdf_tex} 
  	\caption{ideal slope (dashed); measurement (solid). Slope of (a) $\pm 1  i_0$ and (b) $\pm 3 i_0$.}
  	\label{fig:meas_Output_CLoad_100M_1io_3io}
	\end{scriptsize}
\end{figure}
% -----------------------------------------------------------------------------
%%
%% ------------------------------------------------------------------------------
%% !! identical notation regarding color of the measurement data and simulation 			 results!! Also the appearance!!
%% !! technical regarding -> 3i0 do not scale with 3 times 1i0 !
%% !! corresponding simulation data !!

The notation of both figures is the same, the solid black line represents the measured time-domain signal for the frequency of 100 MHz, while the dashed line represents the ideal signal waveform.

% ------------------------------------------------------------------------------
\section{Conclusion}
\label{sec:conclusion}
% ------------------------------------------------------------------------------
A first prototype of a Riemann Pump in \gls{gan} technology is presented to validate the concept of this current steering topology.
The measurement results show the feasibility of synthesizing arbitrary waveforms at 100 MHz.
The potential of the chosen technology promises to cover even higher frequencies to satisfy the condition for the new mobile communication standards.

% ------------------------------------------------------------------------------
% Appendices
% ------------------------------------------------------------------------------

\section*{Acknowledgement}
The authors would like to thank Dirk Meder at Fraunhofer IAF for the assistance with the assembly process.

% ------------------------------------------------------------------------------
% References
% ------------------------------------------------------------------------------

\bibliographystyle{IEEEtran}
\bibliography{IEEEabrv,refs}
\nocite{*}

\end{document}

% EOF
